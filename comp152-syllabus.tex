\documentclass[10pt]{article}
\usepackage{amsmath}
\usepackage{setspace}
\usepackage{hyperref}
\usepackage{booktabs}
\usepackage{longtable}

\setlength{\textheight}{9in} \setlength{\topmargin}{-.5in}
\setlength{\textwidth}{6.5in} \setlength{\oddsidemargin}{0in}
\setlength{\evensidemargin}{0in}

\title{Syllabus \\ COMP 152 \\ Data Structures and Algorithms}
\author{  }
\date{Spring 2022}

\begin{document}
\maketitle

\section{Logistics}
\begin{itemize}
\item \textbf{Where: }
\begin{itemize}
\item Class: Center for Science and Business (CSB), Room 323
\item Lab: Center for Science and Business (CSB), Room 323
\end{itemize}
\item \textbf{When: }
\begin{itemize}
  \item Class: MWF 8--8:50am
  \item Lab: Tu 2--3:50pm
\end{itemize}
\item \textbf{Instructor: } Logan Mayfield
\begin{itemize}
\item \textit{Office: } Center for Science and Business (CSB), Room 344
\item \textit{Phone: } 309-457-2200 % chktex 8
\item \textit{Website: } \url{http://jlmayfield.github.io/}
\item \textit{Email: } lmayfield \textit{at} monmouthcollege \textit{dot} edu
\item \textit{Office Hours: }  MW 2-3pm, Th 9:30-10:30am. By appointment.
\end{itemize}
\item \textbf{Website: } \url{http://jlmayfield.github.io/teaching/COMP152/}
\item \textbf{Credits: } 1 course credit
\end{itemize}
\emph{Note: This Syllabus is subject to change based on specific class needs. Significant deviations from the syllabus will be discussed in class.}


\section{Description, Content, and Learning Goals}

A continuation of COMP 151 that explores the essential data structures
and algorithms of modern computing, including lists, stacks, queues,
heaps, and trees. Students will design, analyze, and build Python
programs that implement and utilize these data structures to solve
computational problems, including a thorough survey of sorting and
search algorithms. These theoretical constructs are complemented by
exposure to good software development practices, including data
abstraction via abstract data types and object-oriented software
design. Strong emphasis is put on analyzing and evaluating how
implementation choices made by the programmer impact overall program
performance and maintainability.
\subsection{Textbook}
\label{sec:orgc3d87e3}

\noindent
Goodrich, Michael T. and Tamassia, Robert and Goldwasser, Michael H. \underline{Data Structures and Algorithms in Python}. Wiley. 2013. ISBN-13: 978-1-118-29027-9.


\subsection{Topics}
\label{sec:org302b48c}
The course aims to cover chapters 1-8, parts of chapters 9-10,
chapter 12, and parts of chapter 14 of the text. This includes, but is
not limited to:

\begin{center}
\begin{tabular}{ll}
\(\bullet\) Review of Python fundamentals & \(\bullet\) Object-Oriented Design Basics and Patterns\\
\(\bullet\) Experimental and Asymptotic Algorithm Analysis & \(\bullet\) Recursion\\
\(\bullet\) Arrays and Sequence Data Types & \(\bullet\) Stacks, Queues, Deques, and Priority Queues\\
\(\bullet\) Abstract Data Types & \(\bullet\) Linked Data Structures\\
\(\bullet\) Binary Search Trees and Tree Traversal Algorithms & \(\bullet\) Maps and Dictionaries\\
\(\bullet\) Sorting, Searching, and Selection Algorithms & \(\bullet\) Graph Algorithms (as time allows)\\
\end{tabular}
\end{center}

\subsection{Programming Environment}
\label{sec:org16c7836}
This course uses \url{https://replit.com/} for all programming assignments. The site's educational and collaborative coding environments allows group work and real-time debugging help from your
instructor to happen whether we're in-person or remote.

\subsection{Student Learning Outcomes}
\label{sec:org99410f2}
This course covers a variety of knowledge areas as categorized by the
ACM/IEEE-CS Task Force on Computing Curricula. Note that not all of
these areas are introduced in this course; some are touched upon
previously and others will be developed further in later courses. At
the end of the course, students should achieve the following learning
outcomes with the specific level of mastery:

\small
\begin{longtable}{|p{1.5in}|p{5.25in}|}
\hline
Knowledge Unit & Learning Outcome with Level of Mastery\\
\hline
\endfirsthead
\multicolumn{2}{l}{Continued from previous page} \\
\hline

Knowledge Unit & Learning Outcome with Level of Mastery \\

\hline
\endhead
\hline\multicolumn{2}{r}{Continued on next page} \\
\endfoot
\endlastfoot
\hline
Basic Analysis & \(\bullet\) In the context of specific algorithms, identify the characteristics of data and/or other\\
 & conditions or assumptions that lead to different behaviors. [Familiarity]\\
 & \(\bullet\) Explain what is meant by "best", "average", and "worst" case behavior\\
 & of an algorithm. [Familiarity]\\
 & \(\bullet\) In the context of specific algorithms, identify the characteristics of\\
 & data and/or other conditions or assumptions that lead to different behaviors.\\
 & \(\bullet\) Perform empirical studies to validate hypotheses about runtime stemming from\\
 & mathematical analysis. Run algorithms on various input sizes and compare performance. [Assessment]\\
 & \(\bullet\) Determine informally the time and space complexity of simple algorithms. [Usage]\\
 & \(\bullet\) Understand the formal definition of big O. [Familiarity]\\
 & \(\bullet\) List and contrast standard complexity classes. [Familiarity]\\
 & \(\bullet\) Give examples that illustrate time-space trade-offs of algorithms. [Familiarity]\\
 & \(\bullet\) Use big O notation formally to give asymptotic upper bounds on time and space complexity of algorithms. [Usage]\\
\hline
Algorithmic Strategies & \(\bullet\) Have facility mapping pseudocode to implementation, implementing examples of\\
 & algorithmic strategies from scratch, and applying them to specific problems. [Familiarity]\\
 & \(\bullet\) Use a divide-and-conquer algorithm to solve an appropriate problem. [Usage]\\
\hline
Fundamental Data & \(\bullet\) Implement simple search algorithms and explain the differences in their time complexities. [Usage, Assessment]\\
Structures and & \(\bullet\) Discuss factors other than computational efficiency that influence\\
Algorithms & the choice of algorithms, such as programming time, maintainability,\\
 & and the use of application-specific patterns in the input data. [Familiarity]\\
 & \(\bullet\) Be able to implement common quadratic and \(O(n \log n)\) sorting algorithms. [Usage]\\
 & \(\bullet\) Demonstrate the ability to evaluate algorithms, to select from a range\\
 & of possible options, to provide justification for that selection, and\\
 & to implement the algorithm in a particular context. [Usage, Assessment]\\
\hline
Fundamentals & \(\bullet\) Explain the concept of modeling and the use of abstraction that\\
 & allows the use of a machine to solve a problem. [Familiarity]\\
\hline
Basics of Counting & \(\bullet\) Perform computations involving modular arithmetic. [Usage]\\
\hline
Processing & \(\bullet\) Analyze simple problem statements to identify relevant information\\
 & and select appropriate processing to solve the problem. [Assessment]\\
 & \(\bullet\) Identify the issues impacting correctness and efficiency of a computation [Familiarity]\\
\hline
Defensive Programming & \(\bullet\) Demonstrate the identification and graceful handling of error conditions. [Usage]\\
\hline
Graphs and Trees & \(\bullet\) Demonstrate different traversal methods for trees [and graphs], including\\
 & pre, post, and in-order traversal of trees. [Usage]\\
 & \(\bullet\) Model a variety of real-world problems in computer science using appropriate\\
 & forms of [graphs and] trees, such as representing \ldots{} a hierarchical file system. [Usage]\\
\hline
Data Modeling & \(\bullet\) Describe the main concepts of the OO model such as object identity,\\
 & type constructors, encapsulation, inheritance, polymorphism, and versioning. [Familiarity]\\
\hline
Object-Oriented & \(\bullet\) Compare and contrast the procedural and object-oriented approach. Understand\\
Programming & both as defining a matrix of operations and variants. [Assessment]\\
 & \(\bullet\) Use subclassing to design simple class hierarchies that allow code to be\\
 & reused for distinct subclasses. [Usage]\\
\hline
Algorithms and Design & \(\bullet\) Determine whether a recursive or iterative solution is most appropriate for a problem. [Assessment]\\
 & \(\bullet\) Implement a divide-and-conquer algorithm for solving a problem. [Usage]\\
 & \(\bullet\) Apply the technique of decomposition to break a problem into smaller pieces. [Usage]\\
 & \(\bullet\) Implement a coherent abstract data type, with loose coupling\\
 & between components and behavior. [Usage]\\
 & \(\bullet\) Discuss the importance of algorithms in the problem-solving process. [Familiarity]\\
 & \(\bullet\) Discuss how a problem may be solved by multiple algorithms, each\\
 & with different properties. [Familiarity]\\
 & \(\bullet\) Implement, test, and debug simple recursive functions. [Usage]\\
\hline
Fundamental & \(\bullet\) Analyze and explain the behavior of simple problems involving the fundamental\\
Programming Concepts & programming constructs. [Assessment]\\
 & \(\bullet\) Identify the base case and the general case of a recursively-defined\\
 & problem. [Assessment]\\
\hline
Fundamental & \(\bullet\) Discuss the appropriate use of built-in data structures. [Familiarity]\\
Data Structures & \(\bullet\) Describe common applications for each data structure in the topic list. [Familiarity]\\
 & \(\bullet\) Write programs that use each of the following data structures: arrays, strings,\\
 & linked lists, stacks, queues, sets, and maps. [Usage]\\
 & \(\bullet\) Compare alternative implementations of data structures with respect to\\
 & performance. [Assessment]\\
 & \(\bullet\) Choose the appropriate data structures for modeling a given problem. [Assessment]\\
\hline
Development Methods & \(\bullet\) Trace the execution of a variety of code segments and write summaries of their\\
 & computations. [Assessment]\\
 & \(\bullet\) Apply a variety of strategies to the testing and debugging of simple\\
 & programs. [Usage]\\
 & \(\bullet\) Construct and debug programs using the standard libraries available with\\
 & the chosen programming language. [Usage]\\
 & \(\bullet\) Apply consistent documentation and program style standards that contribute to\\
 & the readability and maintainability of software.\\
\hline
Software Construction & \(\bullet\) Construct models of the design of a simple software system that are appropriate\\
 & for the paradigm used to design it. [Usage]\\
\hline
Professional Communication & \(\bullet\) Evaluate written technical documentation to detect problems of various kinds. [Assessment]\\
\hline
\end{longtable}
\normalsize

\section{Workload}
% number of/details on midterms, finals, project, homeworks, quizes, etc

The course workload is as follows:
\begin{center}
  \begin{tabular}{ll}
    \underline{Category} & \underline{Number of Assignments} \\
    Exams & 7 \\
    Projects & 2 \\
    Labs & 10 \\
    Homework & 8
  \end{tabular}
\end{center}

\subsection*{Exams}

All exams are weighted equally. There is no midterm or final exam in the sense that the exams are worth more than other exams or that they will necessarily take longer than other exams.  Exams will generally focus on material covered since the previous exam but will be in some sense cumulative due to the nature of programming. Unless stated otherwise, assume that exams will be pencil and paper and that computers will not be available during the exam period.

\subsection*{Projects}

Two larger scale programming projects will be undertaken during the semester. These projects will be individual efforts and will require much more effort than the programs written in lab or as part of homework. Students can expect to have two weeks from the time of the project assignment to complete the project. One or more lab periods will be dedicated to work on the project. It is highly recommend that all students make ample use of the time given on these projects.

\subsection*{Homework}

Homework will consist of exercises from the textbook. Programming problems should be done, tested, and submitted via replit. Non-programming questions can be hand written or typed. All assignments are due at the start of class on the due date.

\subsection*{Labs}

Students will be placed into groups of two or three for each lab. Work will be done using \textit{paired programming} facilitated by replit, a programming practice where each member of the group takes turns typing while the other group member helps look for typos, bugs, and otherwise assists in the design of the code. Each group will submit their work at the end of the lab period regardless of the overall completeness of the assignment. The goal is to make good constructive progress on the assignment. Full credit can and will be given on unfinished work so long as it can be executed to complete some portion of the given task, shows evidence of purposeful progress, and the group made full use of the lab period.

\subsection{Course Engagement Expectations}

The weekly workload for this course will vary by student but on average should be about 13 hours per week.  The follow tables provides a rough estimate of the distribution of this time over different course components.
\begin{center}
\begin{tabular}{ll}
\underline{Assignment Type} & \underline{Time/week} \\
Lectures+Labs       & 6 hours/week \\
Homework          & 1 hours/week \\
Exam Study Time    & 1 hours/week \\
Projects          & 3 hours/week \\
Reading &  2 hours/week \\
\bottomrule
 & 13 hours/week
\end{tabular}
\end{center}


\section{Grades}

This course uses a standard grading scale where percentage grades translate to letter grades as follows:

\begin{center}
\begin{small}
\begin{tabular}{lcl}
\underline{Score} & & \underline{Grade} \\
94--100 & & A \\
90--93 & & A- \\
88--89 & & B+ \\
82--87 & & B \\
80--81 & & B- \\
78--79 & & C+ \\
72--77 & & C \\
70--71 & & C- \\
68--69 & & D+ \\
62--67 & & D \\
60--61 & & D- \\
0--59 & & F
\end{tabular}
\end{small}
\end{center}


Students are always welcome to challenge a grade that they feel is unfair or calculated incorrectly.  Mistakes made in the student's favor will never be corrected to lower a grade.  Mistakes not in the student's favor favor will be corrected.  \textit{Basically, after the initial grading, a score can only go up as the result of a challenge.}



\subsection{Grade Weights}

The final grade is based on a weighted average of particular assignment categories.  Students should be able to estimate your current grade based on your scores and these weights, but you may always visit the instructor \textit{outside of class time} to discuss your current standing and check on some or all of the current course grade.

\begin{center}
  \begin{tabular}{ll}
  \underline{Category} & \underline{Weight} \\
    Exams & 49\% \\ %7 each
    Projects & 20\% \\ %10 each
    Homework & 16\% \\ %2 each
    Labs & 10\% \\ %1 each
    Participation & 5\%
  \end{tabular}
\end{center}

\subsection{Participation, Attendance, \& Late Assignments}

%This course will make almost daily use of Socrative for in-class question and answer sessions. Questions will cover portions of the text that were assigned as reading and will range from simple checks to see if the reading was done to more challenging questions that follow from a close examination of the reading.  For the most part, the only requirement is to provide an answer to every question and participate in the resultant discussions.  On occasion, questions will be evaluated for their correctness and performance on these questions will also factor into the course participation grade.  Students who do the reading and start the homework as soon as possible will have very little to worry about.

%While there is no strict attendance policy, the course participation grade is based in large part on engagement with socrative. Absent students cannot participate in socrative sessions.  Students should avoid unexcused absences, as defined in the college-wide absence policy. Whenever possible, let the instructor know of the absence before it occurs. When unexcused absences do occur, it is the student's responsibility to make up for the lost class time and to seek the permission of the instructor to hand-in or complete assignments that are late due to an unexcused absence.

As a rule, assignments are due at the specified time and no late assignments will be accepted unless an extension was requested prior to the due date. There are, of course, exceptions to this rule and students needing extra time can always contact the instructor for an extension. Do not just give up and eat a zero for the assignment. Ever. There is no penalty in asking for an extension nor is there a limit on extensions.  That being said, there is no guarantee an extension will be given without legitimate need.

This course is designed around the assumption that students \textit{engage in new ideas before they're covered in class meetings}.  This means doing assigned reading, taking a stab at homework problems, and as a result coming to class and lab with some understand about a new idea or, just as likely, with a host of questions about something encountered in the reading and homework. Not attending class, skipping lab, and putting off work to the point that an extension is needed are signs that a student isn't holding up their end of the bargain and is not prepared to participate in class.

\subsection{Academic Honesty}

From the Monmouth College Academic Honesty Policy:
\begin{quote}
  ``We view academic dishonesty as a threat to the integrity and intellectual mission of our institution. Any breach of the academic honesty policy - either intentionally or unintentionally - will be taken seriously and may result not only in failure in the course, but in suspension or expulsion from the college. It is each student’s responsibility to read, understand and comply with the general academic honesty policy at Monmouth College, as defined here in the Scots Guide, and to the specific guidelines for each course, as elaborated on the professor’s syllabus.''

  ``The following areas are examples of violations of the academic honesty policy:
  \begin{enumerate}
  \item Cheating on tests, labs, etc;
  \item Plagiarism, i.e., using the words, ideas, writing, or work of another without giving appropriate credit;
  \item Improper collaboration between students, i.e., not doing one’s own work on outside assignments specified as group projects by the instructor;
  \item Submitting work previously submitted in another course, without previous authorization by the instructor.''
  \end{enumerate}

  ``Please note that this list is not intended to be exhaustive.''
\end{quote}

The complete Monmouth College Academic Honesty Policy can be found on the College web page by clicking on ``Student Life'' then on ``Scot’s Guide'' in the navigation bar to the left, then ``Academic Regulations'' in the navigation bar at the left.  Or you can visit the web page directly by typing in this URL: \url{https://ou.monmouthcollege.edu/life/residence-life/scots-guide/academic-regulations.aspx}

In this course, any violation of the academic honesty policy will have varying consequences depending on the severity of the infraction as judged by the instructor. Minimally, a violation will result in an``F'' or 0 points on the assignment in question. Additionally, the student’s course grade may be lowered by one letter grade. In severe cases, the student will be assigned a course grade of ``F'' and dismissed from the class. All cases of academic dishonesty will be reported to the Associate Dean who may decide to recommend further action to the Admissions and Academic Status Committee, including suspension or dismissal. It is assumed that students will educate themselves regarding what is considered to be academic dishonesty, so excuses or claims of ignorance will not mitigate the consequences of any violations.

\section{Accessibility}

Student Success \& Accessibility Services offers FREE resources to assist Monmouth College students with their academic success. Programs include Supplemental Instruction for select classes, Drop-In and appointment tutoring, and individual Academic Coaching. Our office is here to help all students excel academically, since all students can work toward better grades, practice stronger study skills, and manage their time better.

If you have a disability or had academic accommodations in high school or another college, you may be eligible for academic accommodations at Monmouth College under the Americans with Disabilities Act (ADA). Monmouth College is committed to equal educational access. To discuss any of the services offered, please call or meet with Robert Crawley, Interim Director of Student Success \& Accessibility Services.  SSAS is located in the new ACE space on the first floor of the Hewes Library, opposite Einstein’s Bros Bagels. They can be reached at 309-457-2257 or via email at: ssas@monmouthcollege.edu.



\section{Calendar}


The overall flow and organization of the course follows from the table below.
\begin{center}
\begin{tabular}{lll}
\underline{Section} & \underline{Chapters Covered} & \underline{Assignments} \\
Preliminaries & 1--2 & Labs 1--2, Homework 1, Exam 1. \\
Math \& Analysis & 3 & Lab 3. Homework 2. Exam 2. \\
Ground Floor & 4--5 & Lab 4--5. Homework 3. Exam 3. \\
Algorithms & 12 & Homework 4. Exam 4. Project 1. \\
ADTs & 6 & Lab 6. Homework 5. Exam 5. \\
Data Structures & 7--8 & Lab 7--8. Homework 6. Exam 6. \\
Data Structures \& Algorithms & 9--11,14 & Lab 9--10. Homework 7--8. Exam 7. Project 2.
\end{tabular}
\end{center}


This plan maps to this semesters calendar as follows. \textit{This calendar is subject to change based on the circumstances of the course.} \textbf{Precise dates and other day-to-day details can be found on the course website.}

\begin{center}
\begin{tabular}{lllll}
\underline{Week} & \underline{Dates} & \underline{Notes} & \underline{Assignments Due} & \underline{Chapter(s)}\\
1 & 1/11 --- 1/14 & &  & 1  \\
2 & 1/17 --- 1/21 &  & Homework 1. Lab 1. & 1,2 \\
3 & 1/24 --- 1/28 & & Lab 2.  & 2,3 \\
4 & 1/31 --- 2/4 & &  Lab 3. Exam 1. & 3 \\
5 & 2/7 --- 2/11 & &  Lab 4. Homework 2. & 4\\
6 & 2/14 --- 2/18 & & Lab 5. Exam 2.  & 4,5 \\
7 & 2/21 --- 2/25 & & Homework 3. & 5,12 \\
8 & 2/28 --- 3/4 &  SPRING BREAK (F) &  Homework 4. Exam 3. Project 1. & 12 \\
& 3/7 --- 3/11 &  SPRING BREAK & \\
9 & 3/14 --- 3/18 & & Lab 6. Exam 4. & 6\\
10 & 3/21 --- 3/25 & & Lab 7. Homework 5. & 6,7\\
11 & 3/28 --- 4/1 & & Lab 8. Exam 5. & 7,8\\
12 & 4/4 --- 4/8 &  & Lab 9 & 8 \\
13 & 4/11 --- 4/15 & EASTER (F) & Lab 10. Homework 6. Exam 6. & 9,10 \\
14 & 4/18 --- 4/22 & EASTER (M). & Project 2 & 9,10 \\
15 & 4/25 --- 4/29 & SCHOLAR'S DAY (Tu). & Homework 7. & 11 \\
16 & 5/2 --- 5/6 & READING DAY (Th) & Homework 8. & 11,14 \\
17 & 5/9 --- 5/11 &  & Exam 7 (Tu,11:30am-2:30pm) &

\end{tabular}
\end{center}

\end{document}
