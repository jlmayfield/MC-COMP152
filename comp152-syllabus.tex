\documentclass[10pt]{article}
\usepackage{amsmath}
\usepackage{setspace}
\usepackage{hyperref}
\usepackage{booktabs}

\setlength{\textheight}{9in} \setlength{\topmargin}{-.5in}
\setlength{\textwidth}{6.5in} \setlength{\oddsidemargin}{0in}
\setlength{\evensidemargin}{0in}

\title{Syllabus \\ COMP 152 \\ Data Structures and Algorithms}
\author{  }
\date{Spring 2020}

\begin{document}
\maketitle

\section{Logistics}
\begin{itemize}
\item \textbf{Where: }
\begin{itemize}
\item Class: Center for Science and Business (CSB), Room 309
\item Lab: Center for Science and Business (CSB), Room 309
\end{itemize}
\item \textbf{When: }
\begin{itemize}
  \item Class: MWF 9--9:50am
  \item Lab: W 2--4:50pm
\end{itemize}
\item \textbf{Instructor: } Logan Mayfield
\begin{itemize}
\item \textit{Office: } Center for Science and Business (CSB), Room 344
\item \textit{Phone: } 309-457-2200 % chktex 8
\item \textit{Website: } \url{http://jlmayfield.github.io/}
\item \textit{Email: } lmayfield \textit{at} monmouthcollege \textit{dot} edu
\item \textit{Office Hours: }   or by appointment.
\end{itemize}
\item \textbf{Website: } \url{http://jlmayfield.github.io/teaching/COMP152/}
\item \textbf{Credits: } 1 course credit
\end{itemize}
\emph{Note: This Syllabus is subject to change based on specific class needs. Significant deviations from the syllabus will be discussed in class.}


\section{Description, Content, and Learning Goals}


\subsection{Textbook}

\noindent
Goodrich, Michael T., et. al. \textit{Data Structures and Algorithms in Python}. Wiley. Hoboken, NJ. 2013. % chktex 8


\subsection{Software}

The question and answer software \textit{Socrative} will be used on an almost daily basis. Students will need to create a free student account at \url{https://socrative.com} where it is also possible to participate in socrative sessions via the web.  Socrative also provides free iOS and Android apps for students to use. You can find them on the respective app stores or get direct links at \url{https://socrative.com/apps/}.

\section{Workload}
% number of/details on midterms, finals, project, homeworks, quizes, etc

The course workload is as follows:
\begin{center}
  \begin{tabular}{ll}
    \underline{Category} & \underline{Number of Assignments} \\
    Exams & 6 \\
    Projects & 2 \\
    Labs & 10 \\
    Homework & 8
  \end{tabular}
\end{center}

\subsection*{Exams}

All exams are weighted equally. There is no midterm or final exam in the sense that the exams are worth more than other exams or that they will necessarily take longer than other exams.  Exams will generally focus on material covered since the previous exam but will be in some sense cumulative due to the nature of programming. Unless stated otherwise, assume that exams will be pencil and paper and that computers will not be available during the exam period.

\subsection*{Projects}

Two larger scale programming projects will be undertaken during the semester. These projects will be individual efforts and will require much more effort than the programs written in lab or as part of homework. Students can expect to have two weeks from the time of the project assignment to complete the project. One or more lab periods will be dedicated to work on the project. It is highly recommend that all students make ample use of the time given on these projects.

\subsection*{Homework}

Students will be assigned a set of problems from each chapter of the book covered in the course. These problems are meant to guide reading, prepare the student for in class problems, and survey the material covered by the exam. Each student will turn in their own set of solutions.

\subsection*{Labs}

Students will be placed into groups of two or three for each lab. Work will be done using \textit{paired programming}, a programming practice where each member of the group takes turns typing while the other group member helps look for typos, bugs, and otherwise assists in the design of the code. Each group will submit their work at the end of the lab period regardless of the overall completeness of the assignment. The goal is to make good constructive progress on the assignment. Full credit can and will be given on unfinished work so long as it can be executed to complete some portion of the given task, shows evidence of purposeful progress, and the group made full use of the lab period.

\subsection{Course Engagement Expectations}

The weekly workload for this course will vary by student but on average should be about 13 hours per week.  The follow tables provides a rough estimate of the distribution of this time over different course components.
\begin{center}
\begin{tabular}{ll}
\underline{Assignment Type} & \underline{Time/week} \\
Lectures+Labs       & 6 hours/week \\
Homework          & 1 hours/week \\
Exam Study Time    & 1 hours/week \\
Projects          & 3 hours/week \\
Reading &  2 hours/week \\
\bottomrule
 & 13 hours/week
\end{tabular}
\end{center}


\section{Grades}

This course uses a standard grading scale where percentage grades translate to letter grades as follows:

\begin{center}
\begin{small}
\begin{tabular}{lcl}
\underline{Score} & & \underline{Grade} \\
94--100 & & A \\
90--93 & & A- \\
88--89 & & B+ \\
82--87 & & B \\
80--81 & & B- \\
78--79 & & C+ \\
72--77 & & C \\
70--71 & & C- \\
68--69 & & D+ \\
62--67 & & D \\
60--61 & & D- \\
0--59 & & F
\end{tabular}
\end{small}
\end{center}


Students are always welcome to challenge a grade that they feel is unfair or calculated incorrectly.  Mistakes made in the student's favor will never be corrected to lower a grade.  Mistakes not in the student's favor favor will be corrected.  \textit{Basically, after the initial grading, a score can only go up as the result of a challenge.}



\subsection{Grade Weights}

The final grade is based on a weighted average of particular assignment categories.  Students should be able to estimate your current grade based on your scores and these weights, but you may always visit the instructor \textit{outside of class time} to discuss your current standing and check on some or all of the current course grade.

\begin{center}
  \begin{tabular}{ll}
  \underline{Category} & \underline{Weight} \\
    Exams & 48\% \\ %8 each
    Projects & 24\% \\ %12 each
    Homework & 8\% \\ %1 each
    Labs & 10\% \\ %1 each
    Participation & 10\%
  \end{tabular}
\end{center}

\subsection{Participation, Attendance, \& Late Assignments}

This course will make almost daily use of Socrative for in-class question and answer sessions. Questions will cover portions of the text that were assigned as reading and will range from simple checks to see if the reading was done to more challenging questions that follow from a close examination of the reading.  For the most part, the only requirement is to provide an answer to every question and participate in the resultant discussions.  On occasion, questions will be evaluated for their correctness and performance on these questions will also factor into the course participation grade.  Students who do the reading and start the homework as soon as possible will have very little to worry about.

While there is no strict attendance policy, the course participation grade is based in large part on engagement with socrative. Absent students cannot participate in socrative sessions.  Students should avoid unexcused absences, as defined in the college-wide absence policy. Whenever possible, let the instructor know of the absence before it occurs. When unexcused absences do occur, it is the student's responsibility to make up for the lost class time and to seek the permission of the instructor to hand-in or complete assignments that are late due to an unexcused absence.

In general, assignments are due at the specified time and no late assignments will be accepted unless an extension was requested prior to the due date. There are, of course, exceptions to this rule and students needing extra time can always contact the instructor for an extension. Do not just give up and eat a zero for the assignment. Ever. There is no penalty in asking for an extension nor is there a limit on extensions.  That being said, there is no guarantee an extension will be given without legitimate need.

This course is designed around the assumption that students \textit{engage in new ideas before they're covered in class meetings}.  This means doing assigned reading, taking a stab at homework problems, and as a result coming to class and lab with some understand about a new idea or, just as likely, with a host of questions about something encountered in the reading and homework. Not attending class, skipping lab, and putting off work to the point that an extension is needed are signs that a student isn't holding up their end of the bargain and is not prepared to participate in class.

\subsection{Calendar}

\textit{This calendar is subject to change based on the circumstances of the course.} A detailed, day-by-day calendar of reading requirements and expected exam dates can be found on the course website.

\begin{center}
\begin{tabular}{llll}
\underline{Week} & \underline{Dates} & \underline{Assignments Due} & \underline{Chapter(s)}\\
1 & 1/16 --- 1/17 & &   \\
2 & 1/20 --- 1/24 &  &  \\
3 & 1/27 --- 1/31 & &  \\
4 & 2/3 --- 2/7 & &  \\
5 & 2/10 --- 2/14 & &\\
6 & 2/17 --- 2/21 & &  \\
7 & 2/24 --- 2/28 & &  \\
8 & 3/2 --- 3/5 &   SPRING BREAK (F) &\\
& 3/9 --- 3/13 &  SPRING BREAK & \\
9 & 3/16 --- 3/20 &  &\\
10 & 3/23 --- 3/27 & & \\
11 & 3/30 --- 4/3 & & \\
12 & 4/6 --- 4/10 & EASTER (F)  & \\
13 & 4/13 --- 4/17 & EASTER (M). & \\
14 & 4/20 --- 4/24 & SCHOLAR'S DAY (Tu).  & \\
15 & 4/27 --- 5/1 &  &  \\
16 & 5/4 --- 5/8 & READING DAY (Th) & \\
Final's Week &  &  &  \\
\end{tabular}
\end{center}

\end{document}
