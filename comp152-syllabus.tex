\documentclass[10pt]{article}
\usepackage{amsmath}
\usepackage{setspace}
\usepackage{hyperref}
\usepackage{booktabs}
\usepackage{longtable}

\setlength{\textheight}{9in} \setlength{\topmargin}{-.5in}
\setlength{\textwidth}{6.5in} \setlength{\oddsidemargin}{0in}
\setlength{\evensidemargin}{0in}

\title{Syllabus \\ COMP 152 \\ Data Structures and Algorithms}
\author{  }
\date{Spring 2024}

\begin{document}
\maketitle

\section{Logistics}
\begin{itemize}
\item \textbf{Where: }
\begin{itemize}
\item Class: Center for Science and Business (CSB), Room 323
\item Lab: Center for Science and Business (CSB), Room 309
\end{itemize}
\item \textbf{When: }
\begin{itemize}
  \item Class: MWF 8--8:50am
  \item Lab: W 2--3:50pm
\end{itemize}
\item \textbf{Instructor: } Logan Mayfield
\begin{itemize}
\item \textit{Office: } Center for Science and Business (CSB), Room 344
\item \textit{Phone: } 309-457-2200 % chktex 8
\item \textit{Website: } \url{http://jlmayfield.github.io/}
\item \textit{Email: } lmayfield \textit{at} monmouthcollege \textit{dot} edu
\item \textit{Office Hours: }  MWF 9-10am. Tu 1-2pm. Th 10-11am. By appointment.
\end{itemize}
\item \textbf{Website: } \url{http://jlmayfield.github.io/teaching/COMP152/}
\item \textbf{Credits: } 1 course credit
\end{itemize}
\emph{Note: This Syllabus is subject to change based on specific class needs. Significant deviations from the syllabus will be discussed in class.}


\section{Description, Content, and Learning Goals}

A continuation of COMP 151 that explores the essential data structures
and algorithms of modern computing, including lists, stacks, queues,
heaps, and trees. Students will design, analyze, and build Python
programs that implement and utilize these data structures to solve
computational problems, including a thorough survey of sorting and
search algorithms. These theoretical constructs are complemented by
exposure to good software development practices, including data
abstraction via abstract data types and object-oriented software
design. Strong emphasis is put on analyzing and evaluating how
implementation choices made by the programmer impact overall program
performance and maintainability.
\subsection{Textbook}
\label{sec:orgc3d87e3}

\noindent
Goodrich, Michael T. and Tamassia, Robert and Goldwasser, Michael H. \underline{Data Structures and Algorithms in Python}. Wiley. 2013. ISBN-13: 978-1-118-29027-9.


\subsection{Topics}
\label{sec:org302b48c}
The course aims to cover chapters 1-8, parts of chapters 9-11,
chapter 12, and parts of chapter 14 of the text. This includes, but is
not limited to:

\begin{center}
\begin{tabular}{ll}
\(\bullet\) Review of Python fundamentals & \(\bullet\) Object-Oriented Design Basics and Patterns\\
\(\bullet\) Experimental and Asymptotic Algorithm Analysis & \(\bullet\) Recursion\\
\(\bullet\) Arrays and Sequence Data Types & \(\bullet\) Stacks, Queues, Deques, and Priority Queues\\
\(\bullet\) Abstract Data Types & \(\bullet\) Linked Data Structures\\
\(\bullet\) Binary Search Trees and Tree Traversal Algorithms & \(\bullet\) Maps and Dictionaries\\
\(\bullet\) Sorting, Searching, and Selection Algorithms & \(\bullet\) Graph Algorithms (as time allows)\\
\end{tabular}
\end{center}

\subsection{Programming Environment}
\label{sec:org16c7836}
Students should make every effort to setup a programming environment on their personal computer.  If this is not possible
or desirable, then a programming environment will be made available on campus computers. 

\begin{itemize}
\item \textit{Language:} Python - \url{https://www.python.org/downloads/} 
\item \textit{Development Environment:} VS Code \url{https://code.visualstudio.com/}
\item \textit{Version Control and Assignment Management:} Git \url{https://git-scm.com/downloads} and Github Classroom \url{https://classroom.github.com/}
\end{itemize}

The language and version control system are non-negotiable. Everyone must use them. You are welcome to explore 
development environments besides VS Code, but you're likely going to be on your own if you run into installation 
and configuration problems. 

  
\section{Workload}
% number of/details on midterms, finals, project, homeworks, quizes, etc


Time spent on work for this course will likely vary by student and will, in general, vary week to week. On average, this course should require about 13 hours of work per week per student.  The following table provides a rough estimate of the distribution of this time over different course components.
\begin{center}
\begin{tabular}{ll}
\underline{Assignment Type} & \underline{Time/week} \\
Class Time \& Labs       & 5 hours/week \\
Self-Study \& Reading & 2-3 hours/week \\
Projects           & 3 hours/week \\
Homework Problems   & 1-2 hours/week \\
\bottomrule
 & 11-13 hours/week
\end{tabular}
\end{center}


The course uses fairly standard types of assignments: labs, homework problems, exams, projects, etc. The number of such assignments you can expect to complete is given below along with individual assignment type descriptions.
\begin{center}
  \begin{tabular}{ll}
    \underline{Category} & \underline{Number of Assignments} \\
    Projects & 2 \\
    Exams & 5--7 \\    
    Labs & 8--10 \\
    Homework & 4--6 \\
    Portfolio Review \& Self Evaluation Meetings & 4--5 \\
  \end{tabular}
\end{center}

\subsection*{Labs}

Labs are hands on, active learning periods. You'll dive into new ideas and new techniques
through programming problems done with a partner and with the instructor there to help.
Where other assignments are more about reinforcement and assessment of things previously learned,
lab time is a time to dive into new things. \textit{They goal of a lab is to explore, 
to play, to break things then fix them, to ask questions, and to otherwise uncover everything you
can about a new corner of programming.}

Expect to do labs on a weekly basis. They will typically take not more than the two hour lab period
and will never require that you work on them outside of that lab period. We'll typically go over
lab problems in detail during the class period immediately following the lab.

\subsection*{Homework}

\textit{Homework is practice. It's drills. It's activities designed largely to reinforce and strengthen
your understanding of material learned through reading, through class time, and through lab.}
Homework lets you gut check what you think you know and how well you know it. They afford to you the
time to do the work and then check that work using other references, including classmates. They are meant
to be low stakes assignments.  Just don't confuse low stakes with low importance. Practice,
repetition, and reinforcement through homework is important for learning. It is what gets you ready for
exams, projects, and programming outside of this class.

Homework assignments will typically involve completing problems in the online text as well as other
problems related to the current material. The class will work-on and discuss problems of interest between
the assignment and due date of the problem set. A review of homework problems will typically proceed exams.

\subsection*{Exams}

Exams are meant to test your understanding of and ability to apply ideas covered previously in the course.
They are a gut check the current state of your learning. They let you answer the question,
``Do I know the material as well as I think I know the material?''  Compared to homework, they are higher
stakes assessments of your learning because you typically lack the safety net that is your notes, the text,
and other references.

For the most part, exams will be done in class or in lab and will be announced ahead of time in order
for you to prepare.  Expect these exams to take all or most of a class period and involve multiple
questions or a multi-part problem. On rare occasions we may have small pop-exams that are unannounced,
involve one or two quick questions, and will only take up a small portion of the start of a class period.

\subsection*{Projects}

\textit{You should look at the projects like game day or the big performance. They are, in large part,
what we're preparing for with other smaller assignments.} You should give them your best effort and a
great deal of your time. You'll learn and grow the most as a programmer by really digging in and engaging
in the projects and all of the challenges you'll face when working on them.

Projects are large scale programming assignments done over the course of two weeks. They will
draw on everything you've done and learned in the class up to that point. They will also typically
involve some new ideas that you must navigate and integrate into your work. Some lab and class time
will be used to work on the projects, but you should plan for the bulk of your work on them to take
place outside of class and lab.


\subsection*{Portfolio Review \& Self-Evaluation}

Self-reflection and self-evaluation is a critical component of learning and vital to a growth mindset.
We will keep a portfolio of the work you do throughout the semester. Much of this will be done automatically
by our assignment managment and version control software. At regular intervals throughout the semester you will
meet, one-on-one, with me to \textit{present your porfolio}, review items from your porfolio that best 
guage how well you're doing at meeting the course goals and expectations, and discuss how that success maps to 
a letter grade.  



\section{Ungrading \& Final Grades}

This class is largely ungraded. That means your assignments will not be graded for points and your final grade
is not determined by a point-based, numerical grading system. You will get feedback on your work but you will
see points on nothing. You don't earn points for doing work or getting something correct nor do you lose points
for getting something wrong. We're here to learn. Doing the work is how we do that and getting things wrong
some or most of the time is part of learning.

\subsection{Self-Evaluation \& Final Course Grades}

Throughout the semester you'll be asked to engage in regular self-evaluation. This process is described in
detail in additional documentation. Part of the process includes you self-assigning a course grade based on
your self-evaluation. Your self-evaluation and self-assigned grade are then discussed with me in a one-on-one
meeting during which we'll agree upon your current grade. The key here is that \textit{your self-evaluation
and self-assigned grade begins the conversation, not my assigned points.}

Below are some general rules of thumb we'll try to stick to when talking about grades. They relate grades to
course competency expectations and Monmouth College policy.
\begin{itemize}
  \item \textbf{A} - Exceeding course expectations.
  \item \textbf{B} - Meeting and occasionally exceeding course expectations.
  \item \textbf{C} - Meeting course expectations. \textit{This is the minimum grade required to continue on to COMP152. So, a C means you can be successful in a class that builds upon the things learned in this class.}
  \item \textbf{C-} - Mostly meeting course expectations. \textit{This is the minium grade that counts towards a major.}
  \item \textbf{D} - Occasionally meeting course expectations, but mostly not. \textit{Grades in the D range earn credit towards graduation but fall below GPA requirements.}
  \item \textbf{F} - Did not meet course expectations.
\end{itemize}

My hope is that the self-evaluation and self-directed grading process provides a lot of flexibility in terms
of how you can achieve success in this course and meet your grade goals. If you ever have questions or concerns
about self-evaluations and grades, then I'm more more than willing to discuss them with you at any time.

\subsubsection{Participation, Attendance, \& Timely Work}

I do not have strict attendance and deadline policies, per se, but I do have clear expectations. These
expectations are baked into the dispositional attribute of the course competencies. This attribute
includes things like being \textit{professional, responsible, responsive, and self-directed.}

As far as I'm concerned, signing up for this class means you agree to coming to class and lab,
being on time for class and lab, doing assigned work and submitting it on time, and generally participating
in all the class has to offer.  That being said, life happens and people have different priorities.
You might need to miss class or extend a deadline.  So long as you communicate with me about it, as a
professional would with a co-worker, then we won't have a problem. If you simply skip class without
warning, always show up late, or regularly fail to do assigned work in a timely manner, then I expect that
those failures to meet dispositional expectations to be reflected in your self-evaluation.

There is one exception to my ``no grade-based policy'' on assignments and deadlines and that is the
self-evaluations and reflections. The self-evaluation process is critical to this class and in no way
optional. \textbf{If you fail attend the portfolio review meetings or always show up completely un-prepared
then I reserve to give you a final grade of D or lower for the course.} You'll find I can be pretty relaxed
about a lot of other assignments and deadlines, but I draw the line at the self-evaluation process.


\subsection{Academic Honesty}

From the Monmouth College Academic Honesty Policy:
\begin{quote}
  ``We view academic dishonesty as a threat to the integrity and intellectual mission of our institution. Any breach of the academic honesty policy - either intentionally or unintentionally - will be taken seriously and may result not only in failure in the course, but in suspension or expulsion from the college. It is each student’s responsibility to read, understand and comply with the general academic honesty policy at Monmouth College, as defined here in the Scots Guide, and to the specific guidelines for each course, as elaborated on the professor’s syllabus.''

  ``The following areas are examples of violations of the academic honesty policy:
  \begin{enumerate}
  \item Cheating on tests, labs, etc;
  \item Plagiarism, i.e., using the words, ideas, writing, or work of another without giving appropriate credit;
  \item Improper collaboration between students, i.e., not doing one’s own work on outside assignments specified as group projects by the instructor;
  \item Submitting work previously submitted in another course, without previous authorization by the instructor.''
  \end{enumerate}

  ``Please note that this list is not intended to be exhaustive.''
\end{quote}

The complete Monmouth College Academic Honesty Policy can be found on the College web page by clicking on ``Student Life'' then on ``Scot’s Guide'' in the navigation bar to the left, then ``Academic Regulations'' in the navigation bar at the left.  Or you can visit the web page directly by typing in this URL: \url{https://ou.monmouthcollege.edu/life/residence-life/scots-guide/academic-regulations.aspx}

In this course, any violation of the academic honesty policy will have varying consequences depending on the severity of the infraction as judged by the instructor. Minimally, a violation will result in an``F'' or 0 points on the assignment in question. Additionally, the student’s course grade may be lowered by one letter grade. In severe cases, the student will be assigned a course grade of ``F'' and dismissed from the class. All cases of academic dishonesty will be reported to the Associate Dean who may decide to recommend further action to the Admissions and Academic Status Committee, including suspension or dismissal. It is assumed that students will educate themselves regarding what is considered to be academic dishonesty, so excuses or claims of ignorance will not mitigate the consequences of any violations.

\section{Accessibility}

Student Success \& Accessibility Services offers FREE resources to assist Monmouth College students with their academic success. Programs include Supplemental Instruction for select classes, Drop-In and appointment tutoring, and individual Academic Coaching. Our office is here to help all students excel academically, since all students can work toward better grades, practice stronger study skills, and manage their time better.

If you have a disability or had academic accommodations in high school or another college, you may be eligible for academic accommodations at Monmouth College under the Americans with Disabilities Act (ADA). Monmouth College is committed to equal educational access. To discuss any of the services offered, please call or meet with Robert Crawley, Interim Director of Student Success \& Accessibility Services.  SSAS is located in the new ACE space on the first floor of the Hewes Library, opposite Einstein’s Bros Bagels. They can be reached at 309-457-2257 or via email at: ssas@monmouthcollege.edu.



\section{Calendar}

\textit{This calendar is subject to change based on the circumstances of the course.} \textbf{Precise dates and other day-to-day details can be found on the course website.}

\begin{center}
\begin{tabular}{lllll}
\underline{Week} & \underline{Dates} & \underline{Notes} & \underline{Assignments Due} & \underline{Chapter(s)}\\
1 & 1/17 --- 1/19 & & Lab 1. & 1 \\
2 & 1/22 --- 1/26 &  & Lab 2. & 3  \\
3 & 1/29 --- 2/2 & & Lab 3. Hwk 1. & 3,4 \\
4 & 2/5 --- 2/9 & & Exam 1. Lab 4.  & 4 \\
5 & 2/12 --- 2/16 & & Lab 5. Hwk 2. & 5, 2.3 \\
6 & 2/19 --- 2/23 & Exam 2. Lab 6.  & 5,6 & \\
7 & 2/26 --- 3/1 &  &  Hwk 3. Lab 7. Exam 3.  & 6 \\
8 & 3/4 --- 3/8 & & & 7 \\
& 3/11 --- 3/15 &  SPRING BREAK & \\
9 & 3/18 --- 3/22 & & Hwk 4. & 7\\
10 & 3/25 --- 3/29 & EASTER (F) & Project 1. & 8 \\
11 & 4/1 --- 4/5 & EASTER (M) & Lab 8.   & 8,9\\
12 & 4/8 --- 4/12 & & Lab 9 & 9 \\
13 & 4/15 --- 4/19 & & Exam 4. & 12 \\
14 & 4/22 --- 4/26 & SCHOLAR'S DAY (Tu). & Hwk 5. & 12 \\
15 & 4/29 --- 5/3 & & Exam 5. Project 2. & \\
16 & 5/6 --- 5/10 & READING DAY (Th).  & \textbf{EXAM 6. 5/10. 8-11 am} & 
\end{tabular}
\end{center}

\end{document}
